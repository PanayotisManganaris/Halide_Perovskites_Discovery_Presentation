% Created 2022-06-16 Thu 08:24
% Intended LaTeX compiler: pdflatex
\documentclass[10pt, compress]{beamer}
\usepackage[utf8]{inputenc}
\usepackage[T1]{fontenc}
\usepackage{graphicx}
\usepackage{longtable}
\usepackage{wrapfig}
\usepackage{rotating}
\usepackage[normalem]{ulem}
\usepackage{amsmath}
\usepackage{amssymb}
\usepackage{capt-of}
\usepackage{hyperref}
\usepackage{listings}
\institute[Mannodi Group]{\inst{1} Purdue Materials Science and Engineering Mannodi Group}
\mode<beamer>{\usetheme{Warsaw}}
\useoutertheme{miniframes}
\usepackage{natbib}
\usetheme{default}
\author{Panayotis Manganaris\inst{1}}
\date{\today{}}
\title{Statistical Learning for Halide Perovskite Discovery}
\hypersetup{
 pdfauthor={Panayotis Manganaris\inst{1}},
 pdftitle={Statistical Learning for Halide Perovskite Discovery},
 pdfkeywords={},
 pdfsubject={},
 pdfcreator={Emacs 29.0.50 (Org mode 9.5.3)}, 
 pdflang={English}}
\begin{document}

\maketitle
\begin{frame}{Outline}
\tableofcontents
\end{frame}

\section{AI Background}
\label{sec:org6d9fe30}
\begin{frame}[allowframebreaks]{Artificial Intelligence}
\begin{block}{The Four Approached to AI}
\begin{center}
\begin{tabular}{ll}
Thinking Humanly & Thinking Rationally\\
- Turing test approach & - Laws of Thought\\
(The Six Fields of AI) & -- logical positing\\
-- NLP & -- proven algorithms\\
-- Knowledge Representation & -- correct inference\\
-- automated reasoning & -- syllogistic reason\\
-- Machine Learning & \\
-- computer vision & \\
-- robotics & \\
\hline
\hline
Acting Humanly & Acting Rationally\\
- cognitive modeling approach & - The rational agent\\
-- neuromorphic algorithms & -- inference + reflex\\
 & -- inference vs deduction\\
\end{tabular}
\end{center}
\citet{russell-2010-artif}
\end{block}
\end{frame}

\begin{frame}[label={sec:org5f52d2c}]{Machine Learning}
\begin{block}{ML Contributes to AI}
\begin{itemize}
\item Adaptable \alert{agent}
\begin{itemize}
\item Contextual judgment of \alert{percept} relevance
\item Autonomous utilization of \alert{percept sequence}
\end{itemize}
\item Learning
\begin{itemize}
\item \alert{function} performance improves with exposure to more percepts
\end{itemize}
\end{itemize}
\end{block}
\begin{definition}[Artifical Agency]
\begin{description}
\item[{agent}] self-contained sensor->function->action pipeline
\item[{function}] Set of all possible responses for all possible percepts
\item[{percept}] sensory input
\item[{percept sequence}] history of sensory input
\end{description}
\end{definition}
\end{frame}
\begin{frame}[label={sec:orgb72da13}]{Inverse Design}
\begin{block}{A Type of AI Implementation}
\begin{description}
\item[{senses}] maps points in many dimensions
\item[{function}] reliably navigates it's environment searching for optima
\item[{action}] returns its findings to human interpreters
\end{description}
\end{block}
\begin{block}{Train Consequentially}
\begin{enumerate}
\item Minimize Loss
\item Maximize Score
\end{enumerate}
\end{block}
\end{frame}
\section{Chemistry Background}
\label{sec:org67eb681}
\begin{frame}[label={sec:orgc23004d}]{Perovskite Structure and Chemistry}
\begin{figure}[htbp]
\centering
\includegraphics[width=200]{cubic_perovskite.png}
\caption{Example of hybrid organic-inorganic MAPbI\textsubscript{3} \cite{mannodi-kanakkithodi-2022-data-driven}}
\end{figure}
\end{frame}
\begin{frame}[label={sec:org16cacb2}]{Our Dataset}
\begin{columns}
\begin{column}{0.35\columnwidth}
\begin{block}{DFT Simulations}
\begin{enumerate}
\item geometry optimization
\item Static band structure and optical absorption
\end{enumerate}
\end{block}
\begin{block}{Levels of Theory}
\begin{itemize}
\item PBE
\item HSE06
\item PBE+HSE06(SOC)
\item Experimental
\end{itemize}
\end{block}
\end{column}
\begin{column}{0.65\columnwidth}
\begin{center}
\begin{tabular}{lrrl}
Formula & bg\textsubscript{eV} & \(\eta\) & LoT\\
\hline
MAPbCl3 & 3.0300 & 0.0020 & EXP\\
CsPbI0.375Br2.625 & 1.6880 & 0.1532 & PBE\\
RbSnBr2.625Cl0.375 & 1.4467 & NaN & HSE\\
CsGeCl3 & 1.0510 & 0.1767 & PBE\\
MASr0.5Pb0.5Cl3 & 5.3125 & NaN & HSE\\
MABa0.25Pb0.75I3 & 1.9980 & 0.0155 & PBE\\
MASnI3 & 2.5741 & NaN & HSE\\
MACa0.5Pb0.5Cl3 & 5.3219 & NaN & HSE\\
\ldots{} & \ldots{} & \ldots{} & \ldots{}\\
\end{tabular}
\end{center}
\end{column}
\end{columns}
\end{frame}
\begin{frame}[allowframebreaks]{Band Gap Fidelity}
\begin{figure}[htbp]
\centering
\includegraphics[width=250]{pbe_v_hse_bg.png}
\caption{PBE vs HSE Band Gaps}
\end{figure}
\cite{almora-2020-devic-perfor}
\begin{columns}
\begin{column}{0.4\columnwidth}
\begin{figure}[htbp]
\centering
\includegraphics[height=110]{pbe_v_almora_bg.png}
\caption{PBE vs Almora BG}
\end{figure}
\end{column}
\begin{column}{0.6\columnwidth}
\begin{figure}[htbp]
\centering
\includegraphics[height=110]{hse_v_almora_bg.png}
\caption{HSE vs Almora BG}
\end{figure}
\end{column}
\end{columns}
\end{frame}
\section{Pipeline}
\label{sec:org78ce262}
\begin{frame}[label={sec:orgb9045fa}]{Data Pre-Processing}
\begin{figure}[htbp]
\centering
\includegraphics[width=300]{./data_proc.png}
\caption{Data Preprocessing Workflow to Implement with Python Pandas}
\end{figure}
\end{frame}

\begin{frame}[label={sec:org8584cb5}]{Machine Learning Pipeline}
\begin{figure}[htbp]
\centering
\includegraphics[width=300]{./ML_pipe.png}
\caption{Machine Learning Pipelin to Implement with Python SciKit-Learn}
\end{figure}
\end{frame}

\begin{frame}[allowframebreaks]{Implementation in Jupyter Python}
\small
\lstinputlisting[language=Python]{pretty_pipeline.py}
\end{frame}
\section{Reference}
\label{sec:org0bbdce4}
\begin{frame}[allowframebreaks]{Reference}
\bibliographystyle{plainnat}
\bibliography{../../../org/bibliotex/bibliotex}
\end{frame}
\end{document}